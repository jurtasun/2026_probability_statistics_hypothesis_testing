% Chapter4. Introduction to hypothesis testing ........................................................................
\chapter{Introduction to hypothesis testing}
\label{chapter4}

\epigraph{\textit{The object of statistical science is the reduction of data to relevant information.}}{— Ronald A. Fisher}

The term hypothesis testing lies on top of the two pillars we have mentioned in previous chapters. 

\medskip

Once we know the foundations of probability theory, we can make assumptions about the true popuplation parameters, through expected values [...].

Once we have some notions of descipive stastistics, we can make assumptions about the true popuplation parameters, through expected values [...].

\subsection*{Historical Note}
Fisher (1922, 1925) connected least squares, likelihood, and sampling distributions, establishing the foundations of modern inference.
Neyman (1937) formalized confidence intervals as frequentist procedures, contributing to philosophical debates on inference that continue today.


\section{Prediction vs inference revisted}

\section{General approach to hypothesis testing}

\section{Statistical tests: some examples}

\subsection{Compare sample mean with hypothesized value - One sample t-test}

\subsection{Compare sample means of two independent groups - Two sample t-test}

\subsection{Compare variation on two groups - Fisher's exact test}

\subsection{Compare variation o multiple groups - Fisher's ANOVA}

\subsection{Compare distributions and testing for normality - $\chi^{2}$ test}

\section{Parametric and non-parametric tests}

\section{Comparing data and normalization}

\newpage

\subsection*{Exercises}

\textbf{1.} Exercise [...].\\

\textbf{2.} Exercise [...].\\

\textbf{3.} Exercise [...].\\

\newpage

\subsection*{Solutions}

\textbf{1.} Solution [...].\\

\textbf{2.} Solution [...].\\

\textbf{3.} Solution [...].\\
