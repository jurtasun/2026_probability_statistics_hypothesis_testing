\chapter{A review of integral calculus}
\label{appendix3}

Integral calculus is concerned with accumulation and total change. It provides a way to answer questions such as how much area lies under a curve, how far an object travels when its speed changes over time, or how a quantity distributed across a range adds up to a whole. Whereas differential calculus focuses on how quantities change at a single point, integral calculus asks how these small changes combine over an interval. Together, differentiation and integration offer a unified way to reason about continuous change.

\medskip

At an intuitive level, the basic problem of integration is simple: when something varies from place to place or from moment to moment, how can we compute its total effect? Long before modern calculus, mathematicians confronted this problem in concrete settings such as measuring land, volumes, or the paths of moving objects. In ancient Greece, Archimedes developed the method of exhaustion, in which areas and volumes were approximated by adding up many small pieces. Although expressed geometrically rather than algebraically, this approach captured the core idea behind integration: a total quantity can be understood as the limit of increasingly accurate approximations \cite{archimedes_1897_works}.

\medskip

\textbf{Accumulation and integrals:}

In modern calculus, integration formalizes this idea of accumulation. Suppose a quantity changes continuously and its local contribution at position \(x\) is described by a function \(f(x)\). If we divide an interval \([a,b]\) into small pieces of width \(\Delta x\), the total effect can be approximated by adding up the contributions from each piece,
\[
\sum f(x_i)\,\Delta x.
\]
As the pieces are made smaller and more numerous, this approximation becomes more accurate. The definite integral represents the idealized value obtained in the limit,
\[
\int_a^b f(x)\,dx = \lim_{n \to \infty} \sum_{i=1}^n f(x_i)\,\Delta x.
\]
Rather than thinking of this formula technically, it can be read as a precise way of saying: \emph{the whole is obtained by adding up many very small parts}.

\medskip

A fundamental insight of calculus is that accumulation and change are closely connected. The \emph{Fundamental Theorem of Calculus} shows that the total accumulated effect over an interval can be computed using an antiderivative, linking integration directly to differentiation. In practical terms, this means that knowing how a quantity changes locally allows us to determine its overall impact.

\medskip

This viewpoint is especially important in probability and statistics. When a function represents a \emph{probability density}, integration provides the total probability over a range of values. For example, the probability that a continuous random variable lies between \(a\) and \(b\) is obtained by integrating its density over that interval. In this way, integrals translate local likelihoods into global probabilities, making integral calculus a foundational tool for statistical reasoning and inference.

\medskip

\textbf{Fundamental Theorem of Calculus:}

If \(f\) is continuous on \([a,b]\) and \(F\) is an antiderivative of \(f\), then
\[
\int_a^b f(x)\,dx = F(b) - F(a).
\]

\medskip

The systematic development of integral calculus is closely associated with Isaac Newton and Gottfried Wilhelm Leibniz, who independently formulated its fundamental principles in the late $17^{\text{th}}$ century \cite{newton_1687_principia,leibniz_1686_integral}. Leibniz introduced the integral symbol \(\int\), motivated by the idea of summation, a notation that remains standard today.

\medskip

As with differential calculus, rigor was substantially strengthened in the $19^{\text{th}}$ century. Augustin-Louis Cauchy clarified the role of limits and convergence in integration, while Bernhard Riemann introduced a precise definition of the integral based on partitions of intervals, providing a foundation suitable for modern analysis \cite{cauchy_1821_analyse,riemann_1868_werke}.

\medskip

Earlier mathematical traditions also contributed important ideas related to integration. In the Indian mathematical tradition, the Kerala school developed infinite series and summation techniques for computing areas and trigonometric quantities, motivated by astronomical problems \cite{pingree_1978_kerala}. In the medieval Arabic world, mathematicians refined methods for computing areas and volumes and developed geometric techniques closely related to integral reasoning, which later entered Europe through translation movements \cite{berggren_1986_islamic}.

\medskip

From a philosophical standpoint, integral calculus addresses the challenge of understanding the whole as arising from infinitely many parts. By replacing informal geometric intuition with limit-based reasoning, it provides a reliable framework for reasoning about continuous quantities. At an introductory level, integral calculus offers both practical tools and conceptual insight into how local behavior combines to produce global effects \cite{courant_1965_calculus}.
