\chapter{Appendix: A quick review of integral calculus}
\label{appendix3}

Integral calculus is concerned with accumulation, total change, and the measurement of quantities such as area, volume, and total mass. Whereas differential calculus studies how quantities change at a point, integral calculus focuses on how these changes add up over an interval. Together, the two form a unified framework for analyzing continuous phenomena.

\medskip

The basic problem of integration can be stated simply: given a varying quantity, how can one compute its total effect? Early examples include determining the area under a curve or the distance traveled by an object with varying speed. These questions were studied in ancient mathematics, notably by Archimedes, who used geometric arguments to compute areas and volumes with remarkable precision.

\medskip

In modern terms, the integral of a function over an interval is defined as the limit of finite sums, where the interval is subdivided and the contributions of each subinterval are added together. This idea formalizes the intuitive notion of accumulation and provides a bridge between discrete approximation and continuous exactness.

\medskip

A fundamental result of calculus is the \emph{Fundamental Theorem of Calculus}, which establishes a deep connection between differentiation and integration. It shows that integration can be performed by finding an antiderivative, linking the problem of accumulation directly to the study of rates of change. This theorem unifies the two branches of calculus into a single coherent theory.

\medskip

As with differential calculus, the systematic development of integral calculus is attributed to Newton and Leibniz. Leibniz’s notation for integrals, inspired by the idea of summation, remains standard today. Over time, the theory of integration was refined and extended by mathematicians such as Augustin-Louis Cauchy and Bernhard Riemann, who provided precise definitions suitable for rigorous analysis.

\medskip

Philosophically, integral calculus addresses the challenge of understanding the whole from infinitely many parts. By replacing informal geometric reasoning with limit processes, it provides a reliable method for reasoning about continuous quantities. At an introductory level, integral calculus offers both practical tools and conceptual insight into how local behavior combines to produce global effects.
