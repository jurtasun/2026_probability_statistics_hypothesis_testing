\chapter{Calculus I: Core topics on functions and derivatives}
\label{appendix2}

Calculus begins with the study of how quantities depend on one another and how they change. At its foundation lies the concept of a \emph{function}, which formalizes the idea that one quantity is determined by another. Functions provide the language through which variation, motion, and growth are described in mathematics, physics, and the natural sciences.

\medskip

Historically, the notion of a function evolved gradually. Early uses appear implicitly in the work of René Descartes, who introduced coordinate geometry and expressed curves through algebraic equations. The explicit concept of a function as a mapping between quantities was later clarified in the eighteenth century by Leonhard Euler, whose writings established much of the notation and terminology still in use today.

\medskip

A central idea in calculus is that of a \emph{limit}. Limits capture the behavior of a function as its input approaches a given value, even if the function is not defined or not well behaved at that point. Informally, limits allow us to reason about processes that involve approaching, rather than reaching, a value. This concept is essential for making precise sense of continuity, instantaneous change, and accumulation.

\medskip

The derivative arises from the study of limits and provides a precise definition of instantaneous rate of change. Geometrically, the derivative of a function at a point corresponds to the slope of the tangent line at that point. Physically, it describes quantities such as velocity or growth rate. The basic definition of the derivative is given by a limit of difference quotients, linking algebraic computation with geometric intuition.

\medskip

The development of differential calculus is closely associated with Isaac Newton and Gottfried Wilhelm Leibniz, who independently formulated its fundamental principles in the late seventeenth century. Newton emphasized motion and physical interpretation, while Leibniz introduced a symbolic notation that proved especially flexible and influential. Their work laid the groundwork for centuries of mathematical and scientific progress.

\medskip

From a philosophical standpoint, calculus represents an effort to make sense of continuous change using finite reasoning. The introduction of limits resolved long-standing paradoxes about infinity and infinitesimals by replacing informal arguments with precise definitions. Modern calculus, as taught today, builds on this foundation by emphasizing clarity, rigor, and conceptual understanding rather than purely mechanical computation.
