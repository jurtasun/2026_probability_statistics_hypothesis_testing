% Chapter - Introduction to conditional probability ...................................................................
\chapter{Modeling, dependency and correlation}

\epigraph{\textit{The theory of probabilities is at bottom nothing but common sense reduced to calculation.}}{— Pierre-Simon Laplace}

As one might expect, the origins of probability and related concepts can be traced back to very ancient times. Civilizations such as the Babylonians, Egyptians, and Greeks already encountered uncertainty in various aspects of life, including commerce, games of chance, and divination. Consequently, notions of randomness and stochasticity have deep historical roots. For instance, archaeological findings suggest that the earliest known dice date back over 5,000 years, reflecting humanity’s early fascination with chance and unpredictability \cite{finkel2007ancient}. Although these cultures had not yet developed a formal mathematical theory of probability, they recognized recurring patterns in random events and attempted to anticipate outcomes through either empirical observation or superstition. For a detailed historical overview, see Florence Nightingale's 1962 manuscript \textit{"Games, Gods and Gambling"} \cite{david1962games}.

\medskip

While classical Greek and Roman philosophers frequently discussed the nature of chance, necessity, and determinism, their inquiries remained primarily philosophical rather than mathematical. Thinkers such as Cicero distinguished between events occurring by chance and those determined by fate, foreshadowing later developments in probability theory \cite{cicero45bce}. These early ideas, though lacking quantitative formalism, provided the intellectual foundation for later scientific inquiry into randomness and causality.

A significant shift occurred during the late medieval and early Renaissance periods, when more rigorous mathematical ideas began to shape. Italian mathematician and gambler Gerolamo Cardano (1501–1576) made substantial contributions to the mathematical analysis of chance. His work \textit{"Liber de Ludo Aleae"} (\textit{"Book on Games of Chance"}) \cite{cardano1663ludo}, posthumously published in 1663, is one of the earliest known texts to explore probability through the analysis of gambling problems. However, Cardano’s reasoning, while insightful, lacked the symbolic clarity and mathematical rigour of modern probability theory. Readers consulting the original manuscript will notice an ambiguous and sometimes inconsistent symbolic system, quite unlike the formal structures we use nowadays.

\medskip

The formalization of probability as a mathematical discipline did not occur until the 17th century, most notably through the seminal correspondence between Blaise Pascal and Pierre de Fermat. Their work, motivated by problems such as finding a fair division of stakes in interrupted games of chance, introduced foundational concepts such as combinatorics, expected value, and variance \cite{devlin2008unfinished}. These developments paved the way for later contributions by Christiaan Huygens, who in 1657 wrote the first published textbook on probability \textit{"De Ratiociniis in Ludo Aleae"} (\textit{"On Reasoning in Games of Chance"}), and Jacob Bernoulli, whose 1713 \textit{"Ars Conjectandi"} (\textit{"The Art of Conjecturing"}) remains among the most influential early texts in the field. Their works, alongside with many others, collectively laid the groundwork for the probabilistic and statistical methods that foreshadow modern scientific reasoning \cite{huygens1657ratiociniis}, \cite{bernoulli1713ars}, \cite{hald1990history}.

\medskip

The modern axiomatic formulation of probability was introduced in the early 20th century by the Russian mathematician Andrey Kolmogorov. In his 1933 monograph \textit{"Grundbegriffe der Wahrscheinlichkeitsrechnung"} (\textit{"Foundations of the Theory of Probability"}) \cite{kolmogorov1933grundbegriffe}, Kolmogorov synthesized classical and frequentist ideas into a rigorous mathematical framework based on measure theory. His axioms remain the standard foundation for probability theory to this day. It may seem surprising that a concept with such ancient origins was not formally axiomatized until relatively recent times, and we will return to Kolmogorov’s formulation and its implications in greater detail in Chapter 5. Nevertheless, philosophical discussions about the interpretation of probability and its relation to the physical sciences - especially in the context of determinism, epistemology and modern topics such as quantum mechanics - predate Kolmogorov's formulation and continue to evolve to this day.

\section{Linear models}

\section{Polynomial models}

\section{GLMs and some examples}

\newpage

\subsection*{Exercises}

\textbf{1.} Exercise [...].\\

\textbf{2.} Exercise [...].\\

\textbf{3.} Exercise [...].\\

\newpage

\subsection*{Solutions}

\textbf{1.} Solution [...].\\

\textbf{2.} Solution [...].\\

\textbf{3.} Solution [...].\\